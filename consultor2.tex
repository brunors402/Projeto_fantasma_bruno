% Options for packages loaded elsewhere
% Options for packages loaded elsewhere
\PassOptionsToPackage{unicode}{hyperref}
\PassOptionsToPackage{hyphens}{url}
\PassOptionsToPackage{dvipsnames,svgnames,x11names}{xcolor}
%
\documentclass[
  brazilian,
]{estat/estat}
\usepackage{xcolor}
\usepackage[left=3cm,right=2cm,top=3cm,bottom=2cm]{geometry}
\usepackage{amsmath,amssymb}
\setcounter{secnumdepth}{5}
\usepackage{iftex}
\ifPDFTeX
  \usepackage[T1]{fontenc}
  \usepackage[utf8]{inputenc}
  \usepackage{textcomp} % provide euro and other symbols
\else % if luatex or xetex
  \usepackage{unicode-math} % this also loads fontspec
  \defaultfontfeatures{Scale=MatchLowercase}
  \defaultfontfeatures[\rmfamily]{Ligatures=TeX,Scale=1}
\fi
\usepackage{lmodern}
\ifPDFTeX\else
  % xetex/luatex font selection
  \setmainfont[]{Arial}
\fi
% Use upquote if available, for straight quotes in verbatim environments
\IfFileExists{upquote.sty}{\usepackage{upquote}}{}
\IfFileExists{microtype.sty}{% use microtype if available
  \usepackage[]{microtype}
  \UseMicrotypeSet[protrusion]{basicmath} % disable protrusion for tt fonts
}{}
\makeatletter
\@ifundefined{KOMAClassName}{% if non-KOMA class
  \IfFileExists{parskip.sty}{%
    \usepackage{parskip}
  }{% else
    \setlength{\parindent}{0pt}
    \setlength{\parskip}{6pt plus 2pt minus 1pt}}
}{% if KOMA class
  \KOMAoptions{parskip=half}}
\makeatother
% Make \paragraph and \subparagraph free-standing
\makeatletter
\ifx\paragraph\undefined\else
  \let\oldparagraph\paragraph
  \renewcommand{\paragraph}{
    \@ifstar
      \xxxParagraphStar
      \xxxParagraphNoStar
  }
  \newcommand{\xxxParagraphStar}[1]{\oldparagraph*{#1}\mbox{}}
  \newcommand{\xxxParagraphNoStar}[1]{\oldparagraph{#1}\mbox{}}
\fi
\ifx\subparagraph\undefined\else
  \let\oldsubparagraph\subparagraph
  \renewcommand{\subparagraph}{
    \@ifstar
      \xxxSubParagraphStar
      \xxxSubParagraphNoStar
  }
  \newcommand{\xxxSubParagraphStar}[1]{\oldsubparagraph*{#1}\mbox{}}
  \newcommand{\xxxSubParagraphNoStar}[1]{\oldsubparagraph{#1}\mbox{}}
\fi
\makeatother


\usepackage{longtable,booktabs,array}
\usepackage{calc} % for calculating minipage widths
% Correct order of tables after \paragraph or \subparagraph
\usepackage{etoolbox}
\makeatletter
\patchcmd\longtable{\par}{\if@noskipsec\mbox{}\fi\par}{}{}
\makeatother
% Allow footnotes in longtable head/foot
\IfFileExists{footnotehyper.sty}{\usepackage{footnotehyper}}{\usepackage{footnote}}
\makesavenoteenv{longtable}
\usepackage{graphicx}
\makeatletter
\newsavebox\pandoc@box
\newcommand*\pandocbounded[1]{% scales image to fit in text height/width
  \sbox\pandoc@box{#1}%
  \Gscale@div\@tempa{\textheight}{\dimexpr\ht\pandoc@box+\dp\pandoc@box\relax}%
  \Gscale@div\@tempb{\linewidth}{\wd\pandoc@box}%
  \ifdim\@tempb\p@<\@tempa\p@\let\@tempa\@tempb\fi% select the smaller of both
  \ifdim\@tempa\p@<\p@\scalebox{\@tempa}{\usebox\pandoc@box}%
  \else\usebox{\pandoc@box}%
  \fi%
}
% Set default figure placement to htbp
\def\fps@figure{htbp}
\makeatother



\ifLuaTeX
\usepackage[bidi=basic]{babel}
\else
\usepackage[bidi=default]{babel}
\fi
\ifPDFTeX
\else
\babelfont{rm}[]{Arial}
\fi
% get rid of language-specific shorthands (see #6817):
\let\LanguageShortHands\languageshorthands
\def\languageshorthands#1{}


\setlength{\emergencystretch}{3em} % prevent overfull lines

\providecommand{\tightlist}{%
  \setlength{\itemsep}{0pt}\setlength{\parskip}{0pt}}



 


\authors{%
    Estatiano 1 \\
    Estatiano 2\\
    Estatiano 3\\
}

% escreva o nome do cliente aqui
% se for mais de um separe por \\
\client{%
    ESTAT
}
% Baixando pacotes
\RequirePackage{fancyhdr}
\RequirePackage{graphicx}

\setlength\headheight{28pt}  

\setlength{\parindent}{15pt} % Adiciona indentação nos parágrafos
\setlength{\parskip}{0pt} % Adiciona 0 espaço entro os parágrafos

\let\oldsection\section
\renewcommand\section{\clearpage\oldsection}
\makeatletter
\@ifpackageloaded{float}{}{\usepackage{float}}
\floatstyle{plain}
\@ifundefined{c@chapter}{\newfloat{quadro}{h}{loquad}}{\newfloat{quadro}{h}{loquad}[chapter]}
\floatname{quadro}{Quadro}
\floatstyle{plaintop}
\restylefloat{quadro}
\newcommand*\listofquadros{\listof{quadro}{List of Testes}}
\makeatother
\makeatletter
\@ifpackageloaded{caption}{}{\usepackage{caption}}
\AtBeginDocument{%
\ifdefined\contentsname
  \renewcommand*\contentsname{Índice}
\else
  \newcommand\contentsname{Índice}
\fi
\ifdefined\listfigurename
  \renewcommand*\listfigurename{Lista de Figuras}
\else
  \newcommand\listfigurename{Lista de Figuras}
\fi
\ifdefined\listtablename
  \renewcommand*\listtablename{Lista de Tabelas}
\else
  \newcommand\listtablename{Lista de Tabelas}
\fi
\ifdefined\figurename
  \renewcommand*\figurename{Figura}
\else
  \newcommand\figurename{Figura}
\fi
\ifdefined\tablename
  \renewcommand*\tablename{Tabela}
\else
  \newcommand\tablename{Tabela}
\fi
}
\@ifpackageloaded{float}{}{\usepackage{float}}
\floatstyle{ruled}
\@ifundefined{c@chapter}{\newfloat{codelisting}{h}{lop}}{\newfloat{codelisting}{h}{lop}[chapter]}
\floatname{codelisting}{Listagem}
\newcommand*\listoflistings{\listof{codelisting}{Lista de Listagens}}
\captionsetup{labelsep=colon}
\makeatother
\makeatletter
\makeatother
\makeatletter
\@ifpackageloaded{caption}{}{\usepackage{caption}}
\@ifpackageloaded{subcaption}{}{\usepackage{subcaption}}
\makeatother
\usepackage{bookmark}
\IfFileExists{xurl.sty}{\usepackage{xurl}}{} % add URL line breaks if available
\urlstyle{same}
\hypersetup{
  pdflang={pt-br},
  colorlinks=true,
  linkcolor={black},
  filecolor={black},
  citecolor={black},
  urlcolor={black},
  pdfcreator={LaTeX via pandoc}}


\author{}
\date{}
\begin{document}

% Limpando tudo
\fancyhf{} 

% Ajustes do header
\fancyhead[L]{} % limpando o lado esquerdo
\fancyhead[R]{\includegraphics[width=0.20\textwidth]{estat/imagens/estat.png}} % adicionando logo no canto direito
\renewcommand{\headrulewidth}{0pt}   % sem linha embaixo da logo

% Ajustes de fim de página
\fancyfoot[R]{\textcolor{white}{\thepage}} % Número em branco no canto direito

% Aplicando o estilo que acabamos de criar
\pagestyle{fancy} 


\labelformat{quadro}{\textbf{#1}}

\renewcommand*\contentsname{Sumário}
{
\hypersetup{linkcolor=}
\setcounter{tocdepth}{3}
\tableofcontents
}

\section{Objetivo}\label{objetivo}

O propósito é avaliar se existe relação linear entre o peso (em kg) e a
altura (em cm) dos clientes da base de dados, de modo a identificar
padrões antropométricos que possam ser úteis em futuras estratégias de
mercado.

\section{Análises}\label{anuxe1lises}

\subsection{Análise 3}\label{anuxe1lise-3}

Esta análise tem como objetivo verificar a relação entre as variáveis
altura (cm) e peso (kg) dos clientes. Os dados foram extraídos da aba
infos\_clientes do arquivo relatorio\_old\_town\_road.xlsx. Durante o
tratamento, foram realizadas as seguintes transformações:

-Conversão de peso de libras para quilogramas (1 lb = 0.453592 kg);

-Conversão de altura de decímetros para centímetros (1 dm = 10 cm);

-Remoção de valores ausentes (NA) para garantir consistência nas
análises.

O gráfico a seguir ilustra a dispersão entre as duas variáveis, bem como
a linha de tendência linear ajustada.

O \$ref(fig-grafico2) apresenta a dispersão dos clientes em função de
suas alturas e pesos, com uma linha de regressão linear (tracejada)
representando a tendência central.

\begin{figure}[H]

\caption{\label{fig-grafico2}Relação entre altura e peso dos clientes}

\centering{

\pandocbounded{\includegraphics[keepaspectratio]{consultor2_files/figure-pdf/fig-grafico2-1.pdf}}

}

\end{figure}%

\begin{quadro}[H]

\caption{\label{quad-quadro_altura}Medidas resumo da altura}

\centering{

\begin{tabular} { | l |
            S[table-format = 8.2]
            |}
    \hline
        \textbf{Estatística} & \textbf{Valor} \\
        \hline
        Média & 171,48 \\
        Desvio Padrão & 9,87 \\
        Variância & 97,38 \\
        Mínimo & 150 \\
        1º Quartil & 164,8 \\
        Mediana & 171,75 \\
        3º Quartil & 178 \\
        Máximo & 200 \\
    \hline
    \end{tabular}

}

\end{quadro}%

\begin{quadro}[H]

\caption{\label{quad-quadro_peso}Medidas resumo do peso}

\centering{

\begin{quadro}[H]
    \setlength{ \tabcolsep}{9pt}
    \renewcommand{  \arraystretch}{1.20}
    \caption{Medidas resumo da variável Peso (kg)}
    \centering
    \begin{adjustbox}{max width=\textwidth}
    \begin{tabular} { | l |
            S[table-format = 8.2]
            |}
    \hline
        \textbf{Estatística} & \textbf{Valor} \\
        \hline
        Média & 75,19 \\
        Desvio Padrão & 11,92 \\
        Variância & 142 \\
        Mínimo & 45 \\
        1º Quartil & 66,9 \\
        Mediana & 75,3 \\
        3º Quartil & 83,2 \\
        Máximo & 119,3 \\
    \hline
    \end{tabular}

}

\end{quadro}%

De acordo com o @ref(fig-grafico2), observa-se uma correlação positiva
moderada entre o peso e a altura dos clientes, indicando que à medida
que a altura aumenta, o peso tende a aumentar também.

Com base nos resultados obtidos, conclui-se que há relação linear
positiva significativa entre o peso e a altura dos clientes. Em termos
práticos, clientes mais altos tendem a apresentar pesos maiores, o que é
consistente com o comportamento antropométrico esperado. Essas
informações podem auxiliar o cliente Old Town Road Ltda. na segmentação
de mercado e personalização de produtos, considerando perfis físicos
distintos.

Análise Descritiva --- Altura

O \textbf{Quadro}~\ref{quad-quadro_altura} indica uma média de
aproximadamente 171,48 cm, com um desvio-padrão de 9,87 cm, o que sugere
baixa dispersão dos valores em torno da média. A diferença entre o
mínimo e o máximo é pequena, reforçando a homogeneidade da amostra
quanto à estatura. Os quartis revelam que metade dos clientes possui
altura entre 164,8 cm e 178 cm, mostrando uma tendência central bem
definida. Em conjunto com o gráfico de dispersão, essa análise confirma
que indivíduos mais altos tendem a apresentar maiores pesos,
caracterizando uma relação linear positiva significativa entre as
variáveis.

Análise Descritiva --- Peso

O \textbf{Quadro}~\ref{quad-quadro_peso} mostra uma média de
aproximadamente 75,19 kg e um desvio-padrão de 11,92 kg, indicando
moderada variação entre os indivíduos. A amplitude entre o mínimo e o
máximo revela a existência de clientes com perfis físicos distintos,
desde indivíduos mais leves até outros com peso mais elevado. Os quartis
demonstram que 50\% dos clientes possuem peso entre 66,9 kg e 83,2 kg,
caracterizando uma distribuição relativamente equilibrada, sem grandes
assimetrias. Esses resultados estão de acordo com o gráfico de
dispersão, que evidencia uma correlação positiva entre peso e altura.

\begin{table}[H]

\caption{\label{tbl-modalidades}}

\centering{

}

\end{table}%

\(\ref{fig-}\)

{[}@tbl-{]}

{[}**Quadro** @quad-{]}




\end{document}
