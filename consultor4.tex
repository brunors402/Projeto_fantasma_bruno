% Options for packages loaded elsewhere
% Options for packages loaded elsewhere
\PassOptionsToPackage{unicode}{hyperref}
\PassOptionsToPackage{hyphens}{url}
\PassOptionsToPackage{dvipsnames,svgnames,x11names}{xcolor}
%
\documentclass[
  brazilian,
]{estat/estat}
\usepackage{xcolor}
\usepackage[left=3cm,right=2cm,top=3cm,bottom=2cm]{geometry}
\usepackage{amsmath,amssymb}
\setcounter{secnumdepth}{5}
\usepackage{iftex}
\ifPDFTeX
  \usepackage[T1]{fontenc}
  \usepackage[utf8]{inputenc}
  \usepackage{textcomp} % provide euro and other symbols
\else % if luatex or xetex
  \usepackage{unicode-math} % this also loads fontspec
  \defaultfontfeatures{Scale=MatchLowercase}
  \defaultfontfeatures[\rmfamily]{Ligatures=TeX,Scale=1}
\fi
\usepackage{lmodern}
\ifPDFTeX\else
  % xetex/luatex font selection
  \setmainfont[]{Arial}
\fi
% Use upquote if available, for straight quotes in verbatim environments
\IfFileExists{upquote.sty}{\usepackage{upquote}}{}
\IfFileExists{microtype.sty}{% use microtype if available
  \usepackage[]{microtype}
  \UseMicrotypeSet[protrusion]{basicmath} % disable protrusion for tt fonts
}{}
\makeatletter
\@ifundefined{KOMAClassName}{% if non-KOMA class
  \IfFileExists{parskip.sty}{%
    \usepackage{parskip}
  }{% else
    \setlength{\parindent}{0pt}
    \setlength{\parskip}{6pt plus 2pt minus 1pt}}
}{% if KOMA class
  \KOMAoptions{parskip=half}}
\makeatother
% Make \paragraph and \subparagraph free-standing
\makeatletter
\ifx\paragraph\undefined\else
  \let\oldparagraph\paragraph
  \renewcommand{\paragraph}{
    \@ifstar
      \xxxParagraphStar
      \xxxParagraphNoStar
  }
  \newcommand{\xxxParagraphStar}[1]{\oldparagraph*{#1}\mbox{}}
  \newcommand{\xxxParagraphNoStar}[1]{\oldparagraph{#1}\mbox{}}
\fi
\ifx\subparagraph\undefined\else
  \let\oldsubparagraph\subparagraph
  \renewcommand{\subparagraph}{
    \@ifstar
      \xxxSubParagraphStar
      \xxxSubParagraphNoStar
  }
  \newcommand{\xxxSubParagraphStar}[1]{\oldsubparagraph*{#1}\mbox{}}
  \newcommand{\xxxSubParagraphNoStar}[1]{\oldsubparagraph{#1}\mbox{}}
\fi
\makeatother


\usepackage{longtable,booktabs,array}
\usepackage{calc} % for calculating minipage widths
% Correct order of tables after \paragraph or \subparagraph
\usepackage{etoolbox}
\makeatletter
\patchcmd\longtable{\par}{\if@noskipsec\mbox{}\fi\par}{}{}
\makeatother
% Allow footnotes in longtable head/foot
\IfFileExists{footnotehyper.sty}{\usepackage{footnotehyper}}{\usepackage{footnote}}
\makesavenoteenv{longtable}
\usepackage{graphicx}
\makeatletter
\newsavebox\pandoc@box
\newcommand*\pandocbounded[1]{% scales image to fit in text height/width
  \sbox\pandoc@box{#1}%
  \Gscale@div\@tempa{\textheight}{\dimexpr\ht\pandoc@box+\dp\pandoc@box\relax}%
  \Gscale@div\@tempb{\linewidth}{\wd\pandoc@box}%
  \ifdim\@tempb\p@<\@tempa\p@\let\@tempa\@tempb\fi% select the smaller of both
  \ifdim\@tempa\p@<\p@\scalebox{\@tempa}{\usebox\pandoc@box}%
  \else\usebox{\pandoc@box}%
  \fi%
}
% Set default figure placement to htbp
\def\fps@figure{htbp}
\makeatother



\ifLuaTeX
\usepackage[bidi=basic]{babel}
\else
\usepackage[bidi=default]{babel}
\fi
\ifPDFTeX
\else
\babelfont{rm}[]{Arial}
\fi
% get rid of language-specific shorthands (see #6817):
\let\LanguageShortHands\languageshorthands
\def\languageshorthands#1{}


\setlength{\emergencystretch}{3em} % prevent overfull lines

\providecommand{\tightlist}{%
  \setlength{\itemsep}{0pt}\setlength{\parskip}{0pt}}



 


\authors{%
    Estatiano 1 \\
    Estatiano 2\\
    Estatiano 3\\
}

% escreva o nome do cliente aqui
% se for mais de um separe por \\
\client{%
    ESTAT
}
% Baixando pacotes
\RequirePackage{fancyhdr}
\RequirePackage{graphicx}

\setlength\headheight{28pt}  

\setlength{\parindent}{15pt} % Adiciona indentação nos parágrafos
\setlength{\parskip}{0pt} % Adiciona 0 espaço entro os parágrafos

\let\oldsection\section
\renewcommand\section{\clearpage\oldsection}
\makeatletter
\@ifpackageloaded{float}{}{\usepackage{float}}
\floatstyle{plain}
\@ifundefined{c@chapter}{\newfloat{quadro}{h}{loquad}}{\newfloat{quadro}{h}{loquad}[chapter]}
\floatname{quadro}{Quadro}
\floatstyle{plaintop}
\restylefloat{quadro}
\newcommand*\listofquadros{\listof{quadro}{List of Testes}}
\makeatother
\makeatletter
\@ifpackageloaded{caption}{}{\usepackage{caption}}
\AtBeginDocument{%
\ifdefined\contentsname
  \renewcommand*\contentsname{Índice}
\else
  \newcommand\contentsname{Índice}
\fi
\ifdefined\listfigurename
  \renewcommand*\listfigurename{Lista de Figuras}
\else
  \newcommand\listfigurename{Lista de Figuras}
\fi
\ifdefined\listtablename
  \renewcommand*\listtablename{Lista de Tabelas}
\else
  \newcommand\listtablename{Lista de Tabelas}
\fi
\ifdefined\figurename
  \renewcommand*\figurename{Figura}
\else
  \newcommand\figurename{Figura}
\fi
\ifdefined\tablename
  \renewcommand*\tablename{Tabela}
\else
  \newcommand\tablename{Tabela}
\fi
}
\@ifpackageloaded{float}{}{\usepackage{float}}
\floatstyle{ruled}
\@ifundefined{c@chapter}{\newfloat{codelisting}{h}{lop}}{\newfloat{codelisting}{h}{lop}[chapter]}
\floatname{codelisting}{Listagem}
\newcommand*\listoflistings{\listof{codelisting}{Lista de Listagens}}
\captionsetup{labelsep=colon}
\makeatother
\makeatletter
\makeatother
\makeatletter
\@ifpackageloaded{caption}{}{\usepackage{caption}}
\@ifpackageloaded{subcaption}{}{\usepackage{subcaption}}
\makeatother
\usepackage{bookmark}
\IfFileExists{xurl.sty}{\usepackage{xurl}}{} % add URL line breaks if available
\urlstyle{same}
\hypersetup{
  pdflang={pt-br},
  colorlinks=true,
  linkcolor={black},
  filecolor={black},
  citecolor={black},
  urlcolor={black},
  pdfcreator={LaTeX via pandoc}}


\author{}
\date{}
\begin{document}

% Limpando tudo
\fancyhf{} 

% Ajustes do header
\fancyhead[L]{} % limpando o lado esquerdo
\fancyhead[R]{\includegraphics[width=0.20\textwidth]{estat/imagens/estat.png}} % adicionando logo no canto direito
\renewcommand{\headrulewidth}{0pt}   % sem linha embaixo da logo

% Ajustes de fim de página
\fancyfoot[R]{\textcolor{white}{\thepage}} % Número em branco no canto direito

% Aplicando o estilo que acabamos de criar
\pagestyle{fancy} 


\labelformat{quadro}{\textbf{#1}}

\renewcommand*\contentsname{Sumário}
{
\hypersetup{linkcolor=}
\setcounter{tocdepth}{3}
\tableofcontents
}

\section{Caso `receita' não tenha sido criado, gera um dummy só para
evitar
erro}\label{caso-receita-nuxe3o-tenha-sido-criado-gera-um-dummy-suxf3-para-evitar-erro}

if (!exists(``receita'')) \{ receita \textless- data.frame( Ano =
1880:1889, Receita\_Media = rep(0, 10) ) \}

if (!exists(``grafico1'')) \{ grafico1 \textless- ggplot(receita, aes(x
= Ano, y = Receita\_Media)) + geom\_blank() + labs(title = ``Gráfico não
disponível (erro de carregamento)'', x = ``Ano'', y = ``Receita Média
(R\$)'') \}

if (!exists(``grafico2'')) \{ grafico2 \textless- ggplot() +
geom\_blank() + labs(title = ``Gráfico não disponível (erro de
carregamento)'') \}

if (!exists(``grafico4'')) \{ grafico4 \textless- ggplot() +
geom\_blank() + labs(title = ``Gráfico não disponível (erro de
carregamento)'') \}

if (!exists(``grafico\_top3'')) \{ grafico\_top3 \textless- ggplot() +
geom\_blank() + labs(title = ``Gráfico não disponível (erro de
carregamento)'') \}

O objetivo desta análise é identificar os três produtos mais vendidos
nas três lojas com maior receita no ano de 1889, permitindo compreender
o comportamento de consumo e o desempenho das principais unidades da Old
Town Road Ltda.

As variáveis utilizadas foram:

StoreID / NameStore: identificação das lojas.

NameProduct: nome dos produtos comercializados.

Quantity: quantidade vendida.

UnityPrice: preço unitário do produto em dólares, convertido para reais
(cotação de 1 USD = 5,31 BRL).

As seguintes transformações foram aplicadas:

Conversão monetária de USD para BRL.

Filtragem das vendas somente para o ano de 1889.

Cálculo da receita total por loja, seleção das Top 3 lojas e
determinação dos Top 3 produtos mais vendidos em cada uma delas.

A visualização dos resultados foi feita por meio de um gráfico de barras
verticais agrupadas.

\#--- \#title: ``Modelo Projeto - Quarto''

\#output-file: titulo do projeto \#---

\section{Introdução}\label{introduuxe7uxe3o}

O presente relatório tem como objetivo apresentar as principais análises
estatísticas realizadas para a empresa Old Town Road Ltda, sob
consultoria da ESTAT Consultoria Júnior. O projeto tem como finalidade
compreender o comportamento das vendas e do perfil dos clientes no
período de 1880 a 1889, com base nos dados fornecidos pela empresa.

Foram conduzidas quatro análises principais:

Receita média das lojas (1880--1889);

Relação entre peso e altura dos clientes;

Idade dos clientes de Âmbar Seco por loja;

Top 3 produtos mais vendidos nas 3 lojas com maior receita em 1889.

As análises foram realizadas no software R, utilizando o pacote
tidyverse e o padrão de formatação gráfica da ESTAT, definido pela
função theme\_estat(). Todas as variáveis numéricas foram devidamente
tratadas e, quando necessário, transformadas para unidades comparáveis
(como conversão de dólares para reais).

\section{Referencial Teórico}\label{referencial-teuxf3rico}

As análises baseiam-se em conceitos clássicos de estatística descritiva
(média, mediana, quartis, desvio-padrão), análise bivariada (correlação
de Pearson e ajuste linear) e visualização exploratória de dados. A
interpretação das medidas descritivas e das visualizações visa não
apenas descrever os dados, mas oferecer insights acionáveis --- por
exemplo, identificar produtos de alto giro, perfis demográficos
relevantes e possíveis diretrizes para política de estoques e promoções.

\section{Análises}\label{anuxe1lises}

\begin{enumerate}
\def\labelenumi{\arabic{enumi}.}
\tightlist
\item
  Receita média das lojas (1880--1889) O objetivo desta análise é
  avaliar a evolução da receita média das lojas entre os anos de 1880 e
  1889. Foram consideradas todas as lojas registradas no banco de dados
  de vendas e calculada a receita total anual, convertida de dólares
  (USD) para reais (BRL), com cotação de 1 USD = R\$5,31.
\end{enumerate}

As variáveis analisadas são:

Ano (extraído da data da venda),

Receita média por loja (em reais).

A visualização escolhida foi um gráfico de linha, por ser ideal para
acompanhar a evolução temporal da receita média.

O gráfico a seguir apresenta a receita média anual (em R\$) das lojas no
período analisado.

\begin{figure}[H]

\caption{\label{fig-grafico1}Evolução da Receita Média das Lojas
(1880--1889)}

\centering{

\pandocbounded{\includegraphics[keepaspectratio]{consultor4_files/figure-pdf/fig-grafico1-1.pdf}}

}

\end{figure}%

\begin{table}[H]

\caption{\label{tbl-receita_media_ano}Medidas resumo da Receita Média
das Lojas (1880--1889)}

\centering{

  \begin{tabular} { | l |
      S[table-format = 8.2]
    |}
  \hline
  \textbf{Estatística} & \textbf{Valor} \\
  \hline
  Média & 1135,15 \\
  Desvio Padrão & 33,31 \\
  Variância & 1109,79 \\
  Mínimo & 1099,19 \\
  1º Quartil & 1111,75 \\
  Mediana & 1120,97 \\
  3º Quartil & 1162,13 \\
  Máximo & 1196,21 \\
  \hline
  \end{tabular}
  
  

}

\end{table}%

\begin{verbatim}
Observa-se, pela $\ref{fig-grafico1}$, que a receita média anual das lojas apresentou variações moderadas ao longo da década.Nos primeiros anos (1880 a 1883) há uma tendência de estabilidade, seguida por um crescimento gradual até o final do período (1889). Esse comportamento pode estar associado à expansão do comércio regional, aumento da demanda local e melhor desempenho das lojas de maior porte. O padrão observado sugere uma trajetória positiva e consistente, indicando que as operações comerciais da Old Town Road Ltda consolidaram-se nesse intervalo temporal.
\end{verbatim}

A análise evidencia que a receita média anual das lojas aumentou
progressivamente entre 1880 e 1889, refletindo um cenário positivo de
expansão de vendas.

O {[}**Quadro** @quad{]} mostra que a média da receita das lojas entre
1880 e 1889 foi de aproximadamente R\$ 1135,15, com desvio-padrão de R\$
33,31, indicando uma variação moderada entre os anos. A diferença entre
o mínimo e o máximo reflete períodos de baixa e alta sazonalidade,
possivelmente influenciados por flutuações na demanda ou eventos
econômicos locais. A mediana próxima da média sugere uma distribuição
simétrica, sem grandes distorções. Essas medidas complementam o gráfico
anterior ao quantificar a tendência de crescimento gradual e consistente
da receita média ao longo da década.

\begin{enumerate}
\def\labelenumi{\arabic{enumi}.}
\setcounter{enumi}{1}
\tightlist
\item
  Relação entre Peso e Altura dos Clientes Esta análise tem como
  objetivo verificar a relação entre as variáveis altura (cm) e peso
  (kg) dos clientes. Os dados foram extraídos da aba infos\_clientes do
  arquivo relatorio\_old\_town\_road.xlsx. Durante o tratamento, foram
  realizadas as seguintes transformações:
\end{enumerate}

\begin{verbatim}
-Conversão de peso de libras para quilogramas (1 lb = 0.453592 kg);
\end{verbatim}

-Conversão de altura de decímetros para centímetros (1 dm = 10 cm);

-Remoção de valores ausentes (NA) para garantir consistência nas
análises.

O gráfico a seguir ilustra a dispersão entre as duas variáveis, bem como
a linha de tendência linear ajustada.

O \$ref(fig-grafico2) apresenta a dispersão dos clientes em função de
suas alturas e pesos, com uma linha de regressão linear (tracejada)
representando a tendência central.

\pandocbounded{\includegraphics[keepaspectratio]{consultor4_files/figure-pdf/unnamed-chunk-1-1.pdf}}

\begin{quadro}[H]

\caption{\label{quad-quadro_altura}Medidas resumo da altura}

\centering{

\begin{verbatim}
```{=latex}
\begin{tabular} { | l |
    S[table-format = 8.2]
  |}
\hline
\textbf{Estatística} & \textbf{Valor} \\
\hline
Média & 171,48 \\
Desvio Padrão & 9,87 \\
Variância & 97,38 \\
Mínimo & 150 \\
1º Quartil & 164,8 \\
Mediana & 171,75 \\
3º Quartil & 178 \\
Máximo & 200 \\
\hline
\end{tabular}


```
\end{verbatim}

}

\end{quadro}%

\begin{quadro}[H]

\caption{\label{quad-quadro_peso}Medidas resumo da idade\_cliente}

\centering{

        \begin{table}[H]
        \setlength{\tabcolsep}{9pt}
        \renewcommand{\arraystretch}{1.20}
        \caption{Medidas resumo da variável Peso (kg)}
        \centering
        \begin{adjustbox}{max width=\textwidth}
        \begin{tabular} { | l | S[table-format = 8.2] |}
        \hline
        \textbf{Estatística} & \textbf{Valor} \\
        \hline
        Média & 75,19 \\
        Desvio Padrão & 11,92 \\
        Variância & 142 \\
        Mínimo & 45 \\
        1º Quartil & 66,9 \\
        Mediana & 75,3 \\
        3º Quartil & 83,2 \\
        Máximo & 119,3 \\
        \hline
        \end{tabular}
        \end{adjustbox}
        \end{table}
        ```
Medidas resumo do peso
:::
De acordo com o @ref(fig-grafico2), observa-se uma correlação positiva moderada entre o peso e a altura dos clientes, indicando que à medida que a altura aumenta, o peso tende a aumentar também.

Com base nos resultados obtidos, conclui-se que há relação linear positiva significativa entre o peso e a altura dos clientes.
Em termos práticos, clientes mais altos tendem a apresentar pesos maiores, o que é consistente com o comportamento antropométrico esperado.
Essas informações podem auxiliar o cliente Old Town Road Ltda. na segmentação de mercado e personalização de produtos, considerando perfis físicos distintos.


Análise Descritiva — Altura

O [**Quadro** @quad] indica uma média de aproximadamente 171,48 cm,com um desvio-padrão de 9,87 cm, o que sugere baixa dispersão dos valores em torno da média. A diferença entre o mínimo e omáximo é pequena, reforçando a homogeneidade da amostra quanto à estatura.Os quartis revelam que metade dosclientes possui altura entre 164,8 cm e 178 cm,mostrando uma tendência central bem definida.Em conjunto com o gráfico de dispersão, essa análise confirma que indivíduos mais altos tendem a apresentarmaiores pesos, caracterizando uma relação linear positiva significativa entre as variáveis.


Análise Descritiva — Peso

O [**Quadro** @quad] mostra uma média de aproximadamente 75,19 kg e um desvio-padrão de 11,92 kg, indicando moderada variação entre os indivíduos.
A amplitude entre o mínimo e o máximo revela a existência de clientes com perfis físicos distintos,
desde indivíduos mais leves até outros com peso mais elevado.
Os quartis demonstram que 50% dos clientes possuem peso entre 66,9 kg e 83,2 kg,
caracterizando uma distribuição relativamente equilibrada, sem grandes assimetrias.
Esses resultados estão de acordo com o gráfico de dispersão, que evidencia uma correlação positiva entre peso e altura.
        
 3. Idade dos Clientes de Âmbar Seco por Loja
        
O objetivo desta análise é compreender o perfil etário dos clientes da cidade de Âmbar Seco, considerando as diferentes lojas que atuam na região.
        As variáveis envolvidas são:
          
 Age — idade dos clientes (em anos);
 NameStore — nome da loja;
NameCity — nome da cidade (filtrada para “Âmbar Seco”).
        
Os dados foram integrados a partir das planilhas relatorio\_vendas, infos\_lojas, infos\_clientes e infos\_cidades.
As idades foram mantidas em sua escala original (anos completos) e agregadas por loja.
A visualização escolhida foi o boxplot, pois permite observar a dispersão, a mediana e os possíveis valores extremos da idade dos clientes em cada loja.
        

::: {.cell}
::: {.cell-output-display}
![](consultor4_files/figure-pdf/fig-grafico4-1.pdf){#fig-grafico4}
:::
:::

Os dados foram integrados a partir das planilhas `relatorio_vendas`, `infos_lojas`, `infos_clientes` e `infos_cidades`.
As idades foram mantidas em sua escala original (anos completos) e agregadas por loja.
A visualização escolhida foi o boxplot, pois permite observar a dispersão, a mediana e os possíveis valores extremos da idade dos clientes em cada loja.

::: {#quad-idade_cliente layout-align="center" quad-pos="H"}
```{=latex}
\begin{tabular} { | l |
    S[table-format = 2.2]
  |}
\hline
\textbf{Estatística} & \textbf{Valor} \\
\hline
Média & 36,2 \\
Desvio Padrão & 10,15 \\
Variância & 103,11 \\
Mínimo & 15 \\
1º Quartil & 30 \\
Mediana & 35 \\
3º Quartil & 42 \\
Máximo & 80 \\
\hline
\end{tabular}

}

\end{quadro}%

A partir da Figura \ref{fig-grafico4}, observa-se que as lojas da cidade
de Âmbar Seco apresentam diferenças relevantes no perfil etário de seus
clientes. Em todas as lojas, a mediana de idade se concentra
principalmente entre 30 e 40 anos, indicando um público adulto
consolidado economicamente --- algo coerente com um comércio de produtos
do faroeste.

A análise indica que o público das lojas de Âmbar Seco é
predominantemente composto por adultos jovens, com pouca variação entre
os estabelecimentos. Essas informações podem orientar estratégias de
marketing e oferta de produtos mais adequados ao perfil de clientes
dessa cidade.

O {[}**Quadro** @quad{]}da idade dos clientes demonstra que as idades
médias variam entre 30 e 42 ano entre as lojas da cidade de Âmbar Seco,
com desvio-padrão moderado, sugerindo uma clientela heterogênea. Os
valores mínimo e máximo indicam presença de jovens e adultos mais
velhos, reforçando que as lojas atendem públicos de diferentes faixas
etárias. A mediana próxima à média aponta uma distribuição relativamente
simétrica, sem forte concentração em extremos. Essas medidas reforçam as
observações do boxplot, que apontam distribuições equilibradas de idade
entre as lojas.

\begin{enumerate}
\def\labelenumi{\arabic{enumi}.}
\setcounter{enumi}{3}
\tightlist
\item
  Top 3 Produtos Mais Vendidos nas Top 3 Lojas
\end{enumerate}

O objetivo desta análise é identificar os três produtos mais vendidos
nas três lojas com maior receita no ano de 1889, permitindo compreender
o comportamento de consumo e o desempenho das principais unidades da Old
Town Road Ltda.

As variáveis utilizadas foram:

StoreID / NameStore: identificação das lojas.

NameProduct: nome dos produtos comercializados.

Quantity: quantidade vendida.

UnityPrice: preço unitário do produto em dólares, convertido para reais
(cotação de 1 USD = 5,31 BRL).

As seguintes transformações foram aplicadas:

Conversão monetária de USD para BRL.

Filtragem das vendas somente para o ano de 1889.

Cálculo da receita total por loja, seleção das Top 3 lojas e
determinação dos Top 3 produtos mais vendidos em cada uma delas.

A visualização dos resultados foi feita por meio de um gráfico de barras
verticais agrupadas.

\pandocbounded{\includegraphics[keepaspectratio]{consultor4_files/figure-pdf/unnamed-chunk-2-1.pdf}}

\begin{quadro}[H]

\caption{\label{quad-receita_top3}Medidas resumo da Receita Total (Top 3
lojas em 1889)}

\centering{

            \begin{tabular} { | l |
                S[table-format = 8.2]
              |}
            \hline
            \textbf{Estatística} & \textbf{Valor} \\
            \hline
            Média & 191780.6 \\
            Desvio Padrão & 8753.06 \\
            Variância & 76616083 \\
            Mínimo & 181689.1 \\
            1º Quartil & 189014.7 \\
            Mediana & 196340.3 \\
            3º Quartil & 196826.4 \\
            Máximo & 197312.5 \\
            \hline
            \end{tabular}
            
            

}

\end{quadro}%

De acordo com a \(\ref{fig-grafico\_top3}\) , observa-se que as três
lojas com maior receita em 1889 apresentaram padrões distintos de
vendas, mas todas possuem produtos específicos com destaque expressivo
de demanda. Entre as lojas analisadas, nota-se que:

Uma ou duas lojas concentram grande volume em um único produto,
indicando dependência comercial desse item;

Outras apresentam vendas mais equilibradas, o que pode sugerir maior
diversificação no portfólio;

Em termos gerais, há uma concentração nas categorias mais vendidas, o
que pode orientar estratégias de estoque e precificação.

O {[}**Quadro** @quad{]} mostra que as receitas variam
significativamente entre as lojas líderes, reforçando que o desempenho
em vendas não depende apenas do volume de produtos, mas também do preço
médio dos itens e da estrutura de vendas.

Esta análise evidencia os principais produtos responsáveis pelo
faturamento das lojas mais rentáveis da Old Town Road Ltda. no ano de
1889. As informações obtidas podem subsidiar decisões estratégicas
relacionadas à gestão de estoque, planejamento de vendas e foco em
produtos de maior retorno financeiro. A abordagem segue o padrão da
ESTAT, com gráfico institucional e quadro resumo de medidas descritivas.

\section{Conclusões}\label{conclusuxf5es}

Tendência positiva na receita média (1880--1889): o padrão de
crescimento observado indica que a Old Town Road Ltda. experimentou
melhoria de desempenho durante a década analisada. Recomenda-se
investigar causas específicas dos anos com maior crescimento (ex.:
introdução de produtos, campanhas promocionais, expansão de lojas).

Perfil físico dos clientes consistente: a correlação positiva entre
altura e peso, com significância estatística, sugere padrões
antropométricos esperados, informação útil para categorias de produto
sensíveis a medidas corporais (vestuário).

Segmentação etária local: em Âmbar Seco, a clientela tende a se
concentrar em adultos jovens (25--40 anos) ideal para direcionar
campanhas regionais e mix de produtos.

Produtos líderes concentram receita: em 1889, poucas SKUs foram
responsáveis por grande parte do volume nas lojas líderes. A
recomendação operacional imediata é reforçar estoque e promoções desses
itens e avaliar margens para priorização em negociações e sortimento.

Recomendações práticas e próximos passos:

Realizar uma análise de margem por produto (preço × custo) para
identificar os produtos com maior rentabilidade, não apenas volume;

Implementar monitoramento contínuo das top-SKUs por loja e por período
(dashboards no Power BI conforme exigência do projeto);

Desenvolver campanhas segmentadas por faixa etária em municípios-chave
(ex.: Âmbar Seco);

Investigar fatores externos (sazonalidade, eventos) que expliquem picos
observados na receita.

Anexos / Observações técnicas

Todos os gráficos foram produzidos com ggplot2 seguindo theme\_estat()
(paleta e tipografia institucional).

Os quadros resumo foram gerados com print\_quadro\_resumo() (formatação
LaTeX compatível com o template de relatório).




\end{document}
